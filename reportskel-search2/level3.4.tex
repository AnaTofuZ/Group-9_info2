\subsection{Level3.4: 認識率を高める工夫}
\subsubsection{対象とする問題点}
今回対象として考えた問題点は,入力と比べサイズが異なる場合,及び,位置がずれている場合
である。

\subsubsection{改善方法の提案}
上記問題に対して提案された改善方法は,入力データを細かくいくつかのブロックに分け,
"1"が入力されているブロックだけに対してのみ,認識をかけるという方法である。
ブロックに分け,入力のあるブロック(今回のテストデータにおける"1"が
入力されているブロック)に対してのみ相対的な位置やその傾向を用いて認識することで,
サイズが異なる場合や位置のずれに対応できると考えた。また,
今回の改善方法における正確性の向上について,分けるブロック数を増やす,ブロック化した
データをさらにブロック化する(ブロック化を多重に行う)等の提案もされた。

\subsubsection{考察}
今回,対象とした問題は入力データ(今回のテストデータにおける"1"が入力されている部分)
の全体的な位置は変化するものの,"1"が入力されているデータ間の相対的な位置が変化しない
ものであるため,相対的な位置を学習することで認識率を向上させるという方法をとった。
しかし,この方法では入力データにノイズが混入する等の,テストデータ内の
"1"が入力されてるデータ間の相対的な位置が変化するような問題に対応できていない。
 このことから,グループ内の討論では,各問題に対しての改善方法を複数実装した方が
入力データを正しく認識できる確率が向上するという結果に至った。


