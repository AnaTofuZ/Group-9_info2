\subsection{Level3.3: 任意の評価用データを用いた評価}

level3.3では文字認識プログラムnn\_numにおいて,評価用データが教師用データと異なる場合,認識結果に違いが見られるかを実験する.

\subsubsection{アプローチ}
実験を行うにあたり,学習用データと評価用データがずれていくにあたり,認識率がどう変動するか仮説を検討した.
文字認識の場合,計算機に学習させるアルゴリズムによって最終的な認識に至るまでのプロセスが異なると考える.しかし,あくまで認識は学習用データを利用し行う.
教師あり学習の場合,明確な答えが定義されている学習用データそのものが入力された場合の認識率は最大であると考える.
また,今回の実験で使用した文字認識プログラムは0か1の数値データに基づいて学習を行う.
人間では誤差の範囲であるが,計算機の場合1bitの変化で計算した内容が大幅に変化していくのではないかと推測した.
従って,学習用データからずれていくに連れて文字を誤認識する確率が上昇するのではないかとの仮説を立てた.

\subsubsection{結果}

\subsubsection{考察}

