\subsection{Level3.3: 任意の評価用データを用いた評価}

level3.3では文字認識プログラムnn\_numにおいて,評価用データが教師用データと異なる場合,認識結果に違いが見られるかを実験する.

\subsubsection{アプローチ}
実験を行うにあたり,学習用データと評価用データがずれていくにあたり,認識率がどう変動するか仮説を検討した.
文字認識の場合,計算機に学習させるアルゴリズムによって最終的な認識に至るまでのプロセスが異なると考える.しかし,あくまで認識は学習用データを利用し行う.
教師あり学習の場合,明確な答えが定義されている学習用データそのものが入力された場合の認識率は最大であると考える.
また,今回の実験で使用した文字認識プログラムは0か1の数値データに基づいて学習を行う.
人間では誤差の範囲であるが,計算機の場合1bitの変化で計算した内容が大幅に変化していくのではないかと推測した.
従って,学習用データからずれていくに連れて文字を誤認識する確率が上昇するのではないかとの仮説を立てた.

この仮説のもと,「二,三,五」の3種類の数字に着目し,いくつか差分を作成し認識率の変化を測定した.

\subsubsection{実験内容}
作成したデータはそれぞれ漢数字の「二,三,五」を少しずつ変更したものである.
これら3文字はそれぞれ横線が多く,五に関しては横線と斜め,及び縦線で構成されている.
例えば三という文字の二画目の横棒に「払い」の様な記述を加えると五だと認識されるのではないか.
また,二の払いを横棒と認識出来るレベルにまでいれると,一,もしくは三として判断されるのではないかとも考えた.
従って,構成している要素が英数字よりも単純な漢数字の3つで今回は測定を行う.

差分データを作成する場合,機械的に変更するべきであると考える.
しかし今回は文字認識であり,実際には手書き文字などの判断につながっていく分野である.
その為,人間が想定できる範囲の変更を手動で行い,機械的にランダムに変更するアプローチは取らなかった.
実際に作成したデータは
\begin{oframed}
\begin{verbatim}
./nn/nn_num/src/data.level3.3
\end{verbatim}
\end{oframed}
ディレクトリ下に保存している.

\subsubsection{結果}

\subsubsection{考察}

