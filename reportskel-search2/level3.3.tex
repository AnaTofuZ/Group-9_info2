\subsection{Level3.3: 任意の評価用データを用いた評価}

level3.3では文字認識プログラムnn\_numにおいて,評価用データが教師用データと異なる場合,認識結果に違いが見られるかを実験する.

\subsubsection{アプローチ}
実験を行うにあたり,学習用データと評価用データがずれていくにあたり,認識率がどう変動するか仮説を検討した.
文字認識の場合,計算機に学習させるアルゴリズムによって最終的な認識に至るまでのプロセスが異なると考える.しかし,あくまで認識は学習用データを利用し行う.
教師あり学習の場合,明確な答えが定義されている学習用データそのものが入力された場合の認識率は最大であると考える.
また,今回の実験で使用した文字認識プログラムは0か1の数値データに基づいて学習を行う.
人間では誤差の範囲であるが,計算機の場合1bitの変化で計算した内容が大幅に変化していくのではないかと推測した.
従って,学習用データからずれていくに連れて文字を誤認識する確率が上昇するのではないかとの仮説を立てた.

この仮説のもと,「二,三,五」の3種類の数字に着目し,いくつか差分を作成し認識率の変化を測定した.

\subsubsection{実験内容}
作成したデータはそれぞれ漢数字の「二,三,五」を少しずつ変更したものである.
これら3文字はそれぞれ横線が多く,五に関しては横線と斜め,及び縦線で構成されている.
例えば三という文字の二画目の横棒に「払い」の様な記述を加えると五だと認識されるのではないか.
また,二の払いを横棒と認識出来るレベルにまでいれると,一,もしくは三として判断されるのではないかとも考えた.
従って,構成している要素が英数字よりも単純な漢数字の3つで今回は測定を行う.

差分データを作成する場合,機械的に変更するべきであると考える.
しかし今回は文字認識であり,実際には手書き文字などの判断につながっていく分野である.
その為,人間が想定できる範囲の変更を手動で行い,機械的にランダムに変更するアプローチは取らなかった.
しかし,3種類の文字とは言え実験データを用意するのは手間である.
今回はその問題を解決するために,引数に応じて教師用データを10個のテキストファイルへとコピーするシェルスクリプトを作成した.
実際に使用したスクリプトを示す. (ソースコード:\ref{makeeval})
\lstinputlisting[caption=makeeval.sh,label=makeeval]{../nn/nn_num/src/data.level3.3/makeeval.sh}

それ以降の文字の変更は前述の通り手動で行った.

実際に作成したデータは
\begin{oframed}
\begin{verbatim}
./nn/nn_num/src/data.level3.3
\end{verbatim}
\end{oframed}
ディレクトリ下に保存している.

命名規則はそれぞれ,evel [2,3,5]- [0-9]+.txtである.

\subsubsection{変更したデータ群}
今回変更したデータ群を示す.
ただし全件記述すると3*10=30個のデータを記述することになる.
レポートの体裁上,多く記載しすぎても問題があると判断し,今回はそれぞれ3つずつの記載に留めた. (ソースコード:\ref{eval22} $\sim$ \ref{eval510} )
作成したデータは全てディレクトリ内に保存しているので参照されたい.

\begin{figure}[H]
    \begin{center}
        \begin{tabular}{c}

            \begin{minipage}{0.33\hsize}
                \begin{center}
                \lstinputlisting[caption=eval2-1.txt,label=eval21]{../nn/nn_num/src/data.level3.3/eval2-1.txt}
                \end{center}
                \end{minipage}

                \begin{minipage}{0.33\hsize}
                    \begin{center}
                        \lstinputlisting[caption=eval2-3.txt,label=eval23]{../nn/nn_num/src/data.level3.3/eval2-3.txt}
                    \end{center}
                    \end{minipage}
                \begin{minipage}{0.33\hsize}
                    \begin{center}
                        \lstinputlisting[caption=eval2-5.txt,label=eval25]{../nn/nn_num/src/data.level3.3/eval2-5.txt}
                    \end{center}
                    \end{minipage}
        \end{tabular}
\end{center}
\end{figure}

\begin{figure}[H]
    \begin{center}
        \begin{tabular}{c}

            \begin{minipage}{0.33\hsize}
                \begin{center}
                \lstinputlisting[caption=eval3-3.txt,label=eval33]{../nn/nn_num/src/data.level3.3/eval3-3.txt}
                \end{center}
                \end{minipage}

                \begin{minipage}{0.33\hsize}
                    \begin{center}
                        \lstinputlisting[caption=eval3-5.txt,label=eval35]{../nn/nn_num/src/data.level3.3/eval3-5.txt}
                    \end{center}
                    \end{minipage}
                \begin{minipage}{0.33\hsize}
                    \begin{center}
                        \lstinputlisting[caption=eval3-10.txt,label=eval310]{../nn/nn_num/src/data.level3.3/eval3-10.txt}
                    \end{center}
                    \end{minipage}
        \end{tabular}
\end{center}
\end{figure}

\begin{figure}[H]
    \begin{center}
        \begin{tabular}{c}

            \begin{minipage}{0.33\hsize}
                \begin{center}
                \lstinputlisting[caption=eval5-3.txt,label=eval53]{../nn/nn_num/src/data.level3.3/eval5-3.txt}
                \end{center}
                \end{minipage}

                \begin{minipage}{0.33\hsize}
                    \begin{center}
                        \lstinputlisting[caption=eval5-5.txt,label=eval55]{../nn/nn_num/src/data.level3.3/eval5-5.txt}
                    \end{center}
                    \end{minipage}
                \begin{minipage}{0.33\hsize}
                    \begin{center}
                        \lstinputlisting[caption=eval5-10.txt,label=eval510]{../nn/nn_num/src/data.level3.3/eval5-10.txt}
                    \end{center}
                    \end{minipage}
        \end{tabular}
\end{center}
\end{figure}

\subsubsection{結果}
今回実験を行うにあたり,最終的な評価の出力のみ抽出したい.
またテキストファイル化し結果を保存したいので,シェルスクリプトを用いて自動で実験及び結果の抽出を行った.
今回作成し実行したシェルスクリプトを示す. (ソースコード:\ref{searchlevel23})

\lstinputlisting[caption=search\_level2.3.sh,label=searchlevel23]{../nn/nn_num/src/search_level2.3.sh}

このシェルスクリプトではまずdata.level3.3ディレクトリ内のresultがファイル名に含まれるtxtファイルを削除する.
その後ディレクトリ中の評価用テキストファイルを1つずつfor文で変数fileにパスを代入しながら処理を行う.
for文中では,nn\_numを起動する.今回はseed値を1000に固定し3回学習をした後にcheckを行い,fileを評価する.
nn\_numでの一連の処理の結果は一次ファイル.tmpに保存し,得たい評価結果のみcutとsedを用いて切り出す.
切り出した結果は,\${file}.result.txtの命名規則で保存していく.

実際にこのシェルスクリプト実行した結果を示す.

\begin{oframed}
    \begin{verbatim}
$sh search_level2.3.sh
data.level3.3/eval2-1.txt
data.level3.3/eval2-10.txt
data.level3.3/eval2-2.txt
data.level3.3/eval2-3.txt
data.level3.3/eval2-4.txt
data.level3.3/eval2-5.txt
 **中略**
data.level3.3/eval5-4.txt
data.level3.3/eval5-5.txt
data.level3.3/eval5-6.txt
data.level3.3/eval5-7.txt
data.level3.3/eval5-8.txt
data.level3.3/eval5-9.txt\end{verbatim}
\end{oframed}

次にこのシェルスクリプトによって生成された各評価用データの結果を確認していく.
今回も全てのデータを出力するのは冗長だと判断し,先程の例に対しての結果を掲載する.

まず 評価用データ\ref{eval21}に対する出力結果である.

\lstinputlisting[caption=eval2-1.txtresult.txt,label=reval21]{../nn/nn_num/src/data.level3.3/eval2-1.txtresult.txt}

この評価用データはほぼ教師用データと同じであった為か,正しく2であると認識された.
では,続いて横2行を完全に1で埋め尽くした場合,どういった挙動になるかを検証した.

\lstinputlisting[caption=eval2-3.txtresult.txt,label=reval23]{../nn/nn_num/src/data.level3.3/eval2-3.txtresult.txt}

今回もまだ2であると認識された.では,この2に対して払いのような部分を付けた.
\lstinputlisting[caption=eval2-5.txtresult.txt,label=reval25]{../nn/nn_num/src/data.level3.3/eval2-5.txtresult.txt}

今回もまだ2であると認識された.しかしこれら3つを比較していくと,次第に2であるとの認識率が低下し,7が上昇している.
漢数字の七の場合,横線が2つに縦線が入っている.可能性としては,横の2線が強くなり,かつ縦線が増えてきた為七であると認識し始めたと考える.

では,教師用データ2に対して極端に変更した場合を考察する.
今回は右上に縦を揃えず,二つずつ1を配置した.(ソースコード:\ref{eval210})人間では二と読めることが出来る.

\lstinputlisting[caption=eval2-10.txt,label=eval210]{../nn/nn_num/src/data.level3.3/eval2-10.txt}
\lstinputlisting[caption=eval2-10.txtresult.txt,label=reval210]{../nn/nn_num/src/data.level3.3/eval2-10.txtresult.txt}

出力結果 (ソースコード: \ref{reval210})を確認すると,計算機はどの漢数字か判断しかねていることが解る.
これは入力されたデータが極端に小さく,かつ表示されたデータが少ない事が原因ではないかと推測出来る.

\subsubsection{考察}
